We explored an early design and scheduler for STOIC in~\cite{ref:stoic}.  This paper extends this workshop paper with a new scheduling system and consideration of both individual and concurrent edge-cloud placements. In other related works, we consider recent advances in machine learning infrastructure leveraging serverless computing, GPU accelerators, and Kubernetes orchestration service. \cite{ref:serverlessstep} and \cite{ref:berkeleyserverless} conduct a comprehensive survey on serverless computing including challenges and research opportunities. We share the same viewpoint that the use of the serverless execution model will grow for online training and inference applications. \cite{ref:deepserving} provides a prototype for deep learning model serving in a serverless platform. \cite{ref:accelerated} provides another use case for accelerating serverless functions by GPU virtualization in data centers. Unique in our work, STOIC extends an existing serverless framework to support GPU acceleration and distributed function placement across the edge and public clouds. \cite{ref:evaluation} evaluates several serverless frameworks based on Kubernetes. We also employ Kubernetes for container orchestration, which is lightweight, flexible, and developer-friendly.  We concur that Kubernetes is a promising deployment infrastructure for serverless computing.  

The second relevant domain of related work is image recognition on IoT devices and edge clouds. \cite{ref:face} compares the processing time of face recognition between the edge device and IoT, namely smartphones. It concludes that edge devices perform comparably faster and scales better as the number of images increases. We agree with this conclusion, and as such, design STOIC to offload image processing workloads to both edge clouds and public clouds. \cite{ref:DDNN} proposes a distributed deep neural network that allows fast and localized inference at the edge device using truncated layers of a neural network. \cite{ref:cooperative} defines edge cloud offloading as a Markov decision process (MDP) whose objective is to minimize the average processing time per job. Based on this setting, it provides a novel approximate solution to MDP with a one-step policy iteration. These works are complementary to STOIC and we are considering how to incorporate them into the system as part of future work.

Also complimentary to STOIC, are tracing, testing, repair, and profiling tools (which STOIC can leverage) for serverless systems. Multiple works track causal dependencies across distributed serverless deployments for use in optimization, placement, and data repair~\cite{ref:repairdata,deptracing19,gammaray17,aws-xray}. FaaSProfiler~\cite{ref:profile} provides a tool for testing and profiling STOIC as a FaaS platform. \cite{ref:security} proposes a security solution that applies reinforcement learning (RL) to provide secure offloading to the edge nodes to prevent jamming attacks. These related systems can be combined with STOIC to provide a robust serverless ecosystem for distributed IoT devices.
